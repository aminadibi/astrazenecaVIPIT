% interactcadsample.tex
% v1.03 - April 2017

\documentclass[]{interact}

\usepackage{epstopdf}% To incorporate .eps illustrations using PDFLaTeX, etc.
\usepackage{subfigure}% Support for small, `sub' figures and tables
%\usepackage[nolists,tablesfirst]{endfloat}% To `separate' figures and tables from text if required

\usepackage{natbib}% Citation support using natbib.sty
\bibpunct[, ]{(}{)}{;}{a}{}{,}% Citation support using natbib.sty
\renewcommand\bibfont{\fontsize{10}{12}\selectfont}% Bibliography support using natbib.sty

\theoremstyle{plain}% Theorem-like structures provided by amsthm.sty
\newtheorem{theorem}{Theorem}[section]
\newtheorem{lemma}[theorem]{Lemma}
\newtheorem{corollary}[theorem]{Corollary}
\newtheorem{proposition}[theorem]{Proposition}

\theoremstyle{definition}
\newtheorem{definition}[theorem]{Definition}
\newtheorem{example}[theorem]{Example}

\theoremstyle{remark}
\newtheorem{remark}{Remark}
\newtheorem{notation}{Notation}

% see https://stackoverflow.com/a/47122900

% Pandoc citation processing

\usepackage{hyperref}
\usepackage[utf8]{inputenc}
\def\tightlist{}


\begin{document}

\articletype{Preprint}

\title{The Case for Continuing COVID-19 Vaccination of Front-Line
Workers in BC: Benefits Outweight the Risk for Thrombocytopenia}


\author{\name{Amin Adibi$^{a}$}
\affil{$^{a}$Respiratory Evaluation Sciences Program, Faculty of
Pharmaceutical Sciences, University of British Columbia, Vancouver, BC,
Canada}
}

\thanks{CONTACT Amin
Adibi. Email: \href{mailto:amin.adibi@ubc.ca}{\nolinkurl{amin.adibi@ubc.ca}}}

\maketitle

\begin{abstract}
Recently, the National Advisory Committee on Immunization (NACI)
recommended against using the AstraZeneca COVID-19 vaccine pending
further review of the risk for Vaccine-Induced Prothrombotic Immune
Thrombocytopenia (VIPIT). Using straightforward calculations and based
on current evidence, we propose that even if the risk is found to be
causally related to the AstraZeneca vaccine, the benefits of continuing
immunization of essential workers with AstraZeneca by far outweigh the
risk. We consider the case of British Columbia as an example. The
province is expected to received an addtional 246700 doses of
AstraZeneca vaccine through US and COVAX until April 11th, enough to
provide the first dose of vaccine to all unvacinated front-line
workers.We estimate that if British Columbia continues the front-line
worker vaccination program as many as 600 lives could be saved for an
expected mortalilty of only 1 person, even if all essential workers were
under 55 and assuming the highest estimated rate of 1 in 100,000
currently reported for VIPID.
\end{abstract}

\begin{keywords}
COVID19; astrazeneca; vaccination; essentialworkers; clots;
thrombocytopenia; harm-benefit; BC
\end{keywords}

\hypertarget{background}{%
\section{Background}\label{background}}

Recently, NACI recommended against using AstraZeneca COVID-19 Vaccine
for Canadians under the age 55, due to concerns about the incidence of
Vaccine-Induced Prothrombotic Immune Thrombocytopenia (VIPIT) based on
European reports \citep{naci_naci_2021}. On Match 18, 2021, the European
Medicines Agency estimated the incidence of VIPID at approximately 1 per
1,000,000 people vaccinated with the AstraZeneca vaccine
\citep{ema_covid-19_2021}. A higher estimated rate of 1 per 100,000 by
the Paul-Ehrlich Institut in Germany was published on March 19th
\citep{pei_covid-19_2021}. It was this higher rate reported by the
Paul-Ehrlich Institut that led NACI to recommend against using this
vaccine in adults under 55 years old \citep{naci_naci_2021}. BC had
initially slated the AstraZeneca vaccine for outbreak control and
essential workers vaccination program. on March 29th and following
NACI's recommendation, BC paused using the AstraZeneca vaccine for those
under 55 and put the essential workers program on hold.

On April 1st, the UK Medicines \& Healthcare Products Regulatory Agency
updated its own previously reported data to report a total of 22
cerebral venous sinus thrombosis (CVST) and 8 other clot-related events
from 18.1 million doses of the AstraZeneca vaccine (1.66 incidents per
million).

Canadian provinces are expected to receive 1.5 million doses of the
AstraZeneca vaccine through US and another 316,800 doses from the COVAX
program between now and April 11th.

British Columbia expects to receive 246,700 doses from these two
AstraZeneca deliveries, enough to finish providing the first dose to all
remaining essential workers.

The 300,690 doses of Pfizer and 105,900 doses of Moderna vaccines
expected within the same time frame are currently allocated for the
priority groups, indigenous population, and age-based vaccination
campaign currently vaccinating those in their 70s.

\hypertarget{harm-benefit-analysis}{%
\section{Harm-Benefit Analysis}\label{harm-benefit-analysis}}

Assuming that BC allocates all 246,700 doses to essential workers, we
can estimate the expected number of deaths due to VIPID,
\(E(death)_{astrazeneca}\), as shown below. To err on the side of
caution, we assume that each dose of the vaccine is independently
associated with the risk for VIPID, and that all recipients are under 55
and as such at higher risk for VIPID. We also assume that there is
enough uptake that BC is able to administer all these doses.

\[
E(death)_{astrazeneca}  = d \times P(VIPIT|AZ) \times P(death|VIPIT)
\] where \(d\) is the number of doses administered, \(P(VIPIT|AZ)\) is
the risk of VIPIT after receiving each dose, and \(P(death|VIPIT)\) is
the case fatality for VIPIT.

To err on the side of caution, we will follow NACI's lead and assume the
highest reported rate of VIPIT, which is 1 in 100,000 recipients, so
\(P(VIPIT|AZ) = \frac{1}{100,000}\). On the other hand, as reported by
NACI, case fatality due to VIPID is currently estimated at 40\%, but is
likely to decrease as there will be more awareness and better early
treatment. Again to err on the side of caution, we'll keep the estimate
at 40\%: \(P(death|VIPIT)=40\%\)

\[
\begin{aligned}
E(death)_{astrazeneca} & = d \times \frac{1}{100,000} \times \frac{40}{100} \\
& = 246,700 \times \frac{4}{1,000,000} \\
& \approx 1  
\end{aligned}
\] If the second dose of the AstraZeneca vaccine is not associated with
increased and renewed risk for VIPIT, given the current evidence our
best estimate for the number of death associated with AstraZeneca
vaccination campaign for essential workers would be 1. If, however, we
assume that the second dose will pose additional risk of VIPIT,
delivering an additional 300,000 doses of the AstraZeneca vaccine will
likely double that estimate, for an expected mortality of 2 persons in
BC.

In its analysis of AstraZeneca vaccine, NACI weighed the risk of adverse
events against age-stratified risk of mortality due to COVID-19, pending
an overall risk-assessment. However, benefits of the AstraZeneca vaccine
go beyond preventing COVID-related mortality, and include protection
against more common COVID complications in younger adults including
severe disease, hospitalizations, and Long COVID. The recent sharp
decline of COVID-19 cases in the UK suggests that the AstraZeneca
vaccine might also provide protection against onward transmission of the
virus, which could be especially critical in essential workers during
the current wave of COVID cases.

Estimating all benefits of the AstraZeneca vaccine requires transmission
and contact-network modelling. A recently published compartmental
modelling preprint by Mulberry and colleagues at SFU which also
considered potential spread of variants of concern suggests that
vaccinating front-line workers alongside the age-based campaign might
save an additional 600 more lives and prevent as many as 200,000
infections in BC during a 6-8 months period
\citep{mulberry_vaccine_2021}.

\hypertarget{limitations}{%
\section{Limitations}\label{limitations}}

Our analysis is based on currently available estimated rates of 1 in
million to 1 in 100,000 for VIPIT and might need correction should
higher rates of this complication be reported. We also did not consider
the difference in logistics of distributing different vaccines. If, for
example, it is logistically possible to switch the vaccine allocation
for above 55 years old age groups to the AstraZeneca vaccine and use
either Pfizer or Moderna vaccines for younger essential workers without
delay, that might be the preferred approach.

We have also not considered potential sex differences in the risk for
VIPIT. Although cases identified to date have been predominantly female,
it remains unclear whether this was due to more females receiving the
AstraZeneca vaccine or due to an intrinsic difference in risk.

\bibliographystyle{tfcad}
\bibliography{AZVIPIT.bib}




\end{document}
